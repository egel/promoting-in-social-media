%----------------------------------------------------------------------------------------
%	Wprowadzenie
%----------------------------------------------------------------------------------------

\section{Wprowadzenie}
\label{sec:wprowadzenie}

Tu znajdzie się wstęp do naszego dokumentu.

\begin{defn}[Serwis społecznościowy]
bla bla bla definicja
\end{defn}

%-------------------------------------------------------------------------

\clearpage

\section{Facebook}
\label{sec:facebook}
Tu najdzie się duuszoo informacji o fejsie dostarczonych przez Ciebie Łukaszu :).


%------------------------------------------------------------------------
\clearpage

\section{Google+}
\label{sec:google-plus}
Google+ znany również pod innymi nazwami takimi jak \emph{Google Plus} lub \emph{G+}, to serwis społecznościowy będący własnością firmy Google Inc.

Serwis ten podobnie jak Facebook, umożliwia dzielenie się informacjami między użytkownikami sieci poprzez możliwość zamieszczania tekstów, zdjęć, wideoklipów czy linków do innych zasobów w sieci, promując je własną marką lub nazwiskiem.

W sieci krąży bardzo wiele różnych wersji logotypu google+ czy g+, które są często zmieniane i komponowane specjalnie pod kolorystykę i układ skórek witryn internetowych, jednak aby sprezycować rzeczywisty wygląd logotypów przedstawiono na rysunku \ref{fig:logo-google} dwa rodzaje logotypów występujące oficjalnie na stronie Google+ (\url{https://plus.google.com/}).

\begin{figure}[!h]
\centering
\begin{subfigure}{.5\textwidth}
  \centering
  \includegraphics[width=.4\linewidth]{images/googleplus_color.png}
  \caption{Logo tekstowe}
  \label{fig:sub1}
\end{subfigure}%
\begin{subfigure}{.5\textwidth}
  \centering
  \scalebox{0.7}
  {
      \includegraphics[width=.4\linewidth]{images/google-plus-logo.png}
  }  
  \caption{Logo graficzne}
  \label{fig:sub2}
\end{subfigure}
\captionsource{Rodzaje logotypów występujących na witrynie Google+}{\url{https://plus.google.com/}}
\label{fig:logo-google}
\end{figure}

Spośród wszystkich portali społecznościowych Google+ odróżnia się od konkurencji tym, że pragnie stawiać na wiarygodność informacji dostarczanej od użytkowników\footnote{W myśl założeń firmy Google, użytkownika portalu reprezentuje jakaś rzeczywista osoba lub organizacja, przez co uwiarygadnia dany podmiot. Osoby lub firmy nie będące fikcyjnym tworem przesyłając informacje, opinię, uwagę, ect. wnoszą pozytywny wkład w ogólny przekaz informacji, a nie sztuczną bańkę informacyjną ,,produkowaną i przekazywaną dalej'' w sieć, jako zabieg marketyngowy do szybszego i skuteczniejszego wpływania na działania użytkowników.} oraz lokalność.

Umieszczenie informacji, zdjęć, nagrań wideo nie są jedynymi działaniami jakie oferuje g+. Portal pozwala także na tworzenie wydarzeń, opiniowanie produktów czy usług oferowanych przez firmy promujące się w społeczności poprzez rozdawanie plusików, a także intergalny dostęp do innych usług oferowanych przez firmę Google zwiększając tym samym ofertę przystąpienia do społeczności.

%-----------------------------------------

\subsection{Promocja firmy, a Google+}
W chwili obecnej tj. 2 kwartale 2014 roku wyszukiwarka Google zajmuję 1 miejsce wśród narzędzi do wyszykiwania informacji w Polsce (95,59\% udziału rynku), zaraz za nią MSN (2,57\%) oraz Yahoo (0,97\%) \cite{url:gemius-ranking-silnikow-wyszukiwarek}.

Google będąc największym potentatem rozwiązań wyszukiwania informacji w internecie na polskim rynku można pokusić się nawet o stwierdzenie że jest niemal monopolistą rynkowym, spychając rywali na wąski margines.

Tak ogromny udział w rynku zapewnie niemal nieograniczone możliwości kreacji promocji w sieci. Jednak z punktu widzenia wolności i konkurencyjności rynku, ,,wyszukiwarkowy'' baron dyktuje koszty promocji wszystkim tym, którzy korzystają z jego produktów --- a jest to niemal całość polskiego społeczeństwa $\sim$96\%. Tak wielki procent udziałów w rynku, poniekąd zmusza firmy do skorzystania z oferty Googla, jeśli chcą dotrzeć do wiekszości polskich internatów. \\


Jednym z darmowych narzędzi ułatwiających promocję firmy wśród mediów społecznościowych jest wczęśniej wspomniany google+. Google+ jako jedna z usług dostarczanych przez firmę Google ma za zadanie łączyć spółeczność portalu i personalizować ich użytkowników, a przez to uwiarygadniać oraz dzielić się treściami pochodzącymi z wiarygodnego, zaufanego źródła --- przykładowo naszego przyjacjela Pana Michała, który istnieje (nie jest fikcyjną wykreowaną przez media/marketing postacią) i wyraził swoją opinię o jednym z przedmiotów promowanych przez firmę. Dzięki sieci google+ wiem z dużym prawdopodobieństem, że opinia Pana Michała jest wiarygdna, ponieważ go znam (widuję na codzień), a dzięki dobrej opinii o wykonaniu usługi intenieje również większa szansa, że i ja skorzystam z usługi.\\

Ta krótka, lecz ważna notka z punktu widzenia osoby prawnej, w wąskim stopniu przybliża działanie idei portalu oraz istotności potrzeby promocji w nim (a także wszystkich usługach oferowanych przez Google, które poniekąd tworzą jedną spójną całość uzupełniając sie wzajemnie). 

Rozmawiając o usłudze Google+ dyskutujemy tak na prawdę o narzędziu spajającym ze sobą pozostałe usługi dostarczane przez firmę Google w postaci jednego, ułatwiającego zarządzanie Panelu Google+.

%-------------------------------------------

\subsection{Panel Google+}
Panel Google+ to obszar skupiający w jedym miejscu kluczowe informacje dotyczące różnych obszarów istotnych w prowadzeniu firmy w internecie (tu celowo pomijam użytkownika indywidualnego, ponieważ nie jest on tematem dysputy w niniejszej publikacji).\\

Panel Google+ dostarcza m.in:

\begin{itemize}
\item Możliwość aktualizacji danych firmy w 1 miejscu;

\item Narzędzia sprawdzające kompletność i zgodność witryny z wyszukiwarką Google;

\item Wyświetlać wpisy innych użytkowników oraz tworzyć włanse z informacjami, zdjęciami i filmami.

\item Szeroka interakcja z klientami poprzed budowanie precyzyjnego grona odbiorców oraz udoskonalanie swoich zasobów odpowiadając na opinie użytkowników o firmie/usłudze.

\item Rozmawiać bezpośrednio ,,twarzą w twarz'' z klientami dzięki usłudze Google Hangouts (internetowy odpowiednik Skypa lub Vibera);

\item Przeglądać szeroki zakres różnego typu statystyk dotyczących firmy a w tym:
    \begin{itemize}
    \item Najpopularniejsze wyszukiwania na temat firmy w wyszukiwarce Google;
    \item Skąd klienci wyznaczją trasy dojazdu do placówki dzięki usłudze Google Maps;
    \item Sprawdzenie popularności firmy wśród społeczości Google+;
    \end{itemize}

\item Zarządzać reklamami, w tym także poprzez integrację z usługą AdWords Express;

\item Mobilność dostarczanych rozwiązań (Smartphone, Tablet, Komputer);
\end{itemize}

%----------------------------------------------------------------------

\subsection{Możliwości promocji w Google+}
\
Model promocji działalności w Google+ jest dość jasny --- tworząc własny profil lub inaczej stronę (ang. \textit{wall}) umieszczamy wszystkie rzeczy dotyczące firmy tj. informacje oraz mapę dojazdu do firmy, tworzymy posty z atrakcyjnymi ofertami usługami lub produktów, a w zamian uzyskujemy ogromne i w miarę możliwości wiarygodne\footnote{Wiarygodność jest tu ujęta w cudzysłów, ze względu na fakt, iż prawdopodobnie nigdy nie osiągnie poziomu 100\%. Zawsze jakiś odsetek opinii czy uwag na temat produktu będzie przejaskawiony w niegatywną stronę lub po prostu zrobiony z premedytacją (w ramach zagrań nieczystej konkurencji).\\ Jednak ogólne założenia wiarygodności użytkowników w systemie Google+ pozwalają nam z dużym prawdopodobieństwem twierdzić, że opinia danego klienta jest jak najbardziej prawdziwa i wartościowa, tak więc rozsądnie jest brać je wszystkie pod uwagę.} miejsce zwrotów opinni wśród użytkowników, który skorzystali z naszych usług. 

\noindent Dodatkowo dzięki informacji zwrotnej od klientów (posiadających konto Google+) możemy szybko skorygować lub udoskonalić naszą ofertę, a tym samym promując ją dalej wśród znajomych naszych klientów, którzy własnie skorzystali z części naszej oferty i wyrazili opinię. 

To jednak tylko ogólny zarys wstępu do poszczególnych kanałów jakie oferuje Google+. W dalszej części omówimy bardziej szczegółowo poszczególne możliwości usługi w nieco szerszym zakresie, w tym profity jakie niosą ze sobą.


\subsubsection{Wszystko w jednym miejscu}


%----------------------------------------------------------------------------------------
\clearpage
\section{Podsumowanie}

Każdy z portali społecznościowych wyraża nieco odmienne przewodnie idee. 
