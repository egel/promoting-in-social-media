\documentclass[12pt, oneside, a4paper, franc]{report}
\usepackage[OT4, plmath]{polski}                % definicja użycia platex
\usepackage[utf8]{inputenc}                     % kodowanie na UTF-8
\usepackage[OT4]{fontenc}
\usepackage{url}
\title{Projekt i implementacja autorskiego systemu zarządzania treścią}
\author{Maciej Sypień}
\date{\today}


\begin{document}

\begin{abstract}
Dokument ten prezentuje kilka zasad składu tekstu w~systemie \LaTeX. 
\end{abstract}

\chapter{Coś od czego trzeba zacząć}

% pierwsza sekcja
\section{Tekst}\label{sec:tekst}
\LaTeX\ ułatwia autorowi tekstu zarządzanie numerowaniem sekcji, wypunktowaniami oraz odwołaniami do tabel, rysunków i~innych elementów. W~łatwy sposób możemy się odwołać do wzoru \ref{eqn:wzor1}.

% druga sekcja
\section{Matematyka}\label{sec:matematyka}
Poniższy wzór prezentuje możliwości \LaTeX\ w~zakresie składu formuł matematycznych. Wzory są numerowane automatycznie, podobnie jak inne elementy o~których mowa w~sekcji~\ref{sec:tekst}.

\begin{equation}
    E = mc^2,
    \label{eqn:wzor1}
\end{equation}

gdzie

\begin{equation}
    m = \frac{m_0}{\sqrt{1-\frac{v^2}{c^2}}}.
\end{equation}




\end{document}