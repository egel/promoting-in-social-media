\section{Podsumowanie}
\label{sec:podsumowanie}

Media społecznościowe, czyli popularne \textit{social media} w chwili obecnej zaraz po komunikacji mobilnej, są głównym nurtem branży internetowej. Ciągle rosnąca potrzeba dzielenia się informacjami z rodziną, przyjaciółmi, znajomymi, współpracownikami różnymi drogami niemal natychmiast, zaowocowała tym, iż społeczeństwo chętniej i coraz dłużej spędza czas na portalach społecznościowych. 

\noindent Pomocne w tym stają się różnego rodzaju urządzenia mobilne (smartphone, tablet, netbook, palmtop, laptop, ect.). Dzięki tym urządzeniom dostęp do treści jest możliwy z każdego miejsca. Taki obrót sytuacji sprawił, iż liczba osób korzystających z największych i najpopularniejszych serwisów społecznościowych stale i nieustannie rośnie. Sukcesowi takich portali pozazdrościł gigant internetowy firma Google. W drugim półroczy 2011 roku uruchomił serwis Google+.

Usługa Google+ miała różnić się od pozostałych sieci społecznościowych rzetelnością i prawdziwością informacji, w szczególności dla biznesu. Profile w Google+ nie miały wyłącznie dotyczyć sfery biznesu. Ideą profilów jest możliwość komunikacji na szeroką skalę, z każdego urządzenia umożliwiającego komunikację prze Internet.

Drugim serwisem społecznościowym, na polu którego można prowadzić szereg działań marketingowych jest Facebook. Jak wcześniej zostało to już opisane, posiada szereg funkcji, gdzie niektóre z nich zostały zaimplementowane z powodu rosnącego zapotrzebowania na marketing społecznościowy.\\

Telewizja, radio, czy prasa --- tradycyjne media, umożliwiają reklamę danych produktów bardziej jako informację. Ten sam przekaz dla wszystkich. Internet, a w szczególności media społecznościowe dają coś czego tradycyjne mass media nie posiadają, czyli reklamę spersonalizowana, która uczy się ,,rozpoznawać'' klientów, lecz oczywiście nie w sposób dosłowny.

\noindent Firmy kształtują swoje reklamy za pomocą statystyk na stronach profili własnych firm i w ten sposób mogą dostosowywać własny przekaz, by jak najlepiej i jak najpełniej trafił do grupy docelowej, którą dana firma obrała za swój target sprzedażowy. Działania taki miałyby na celu dostosowanie poziomu reklamy i jej przekazu do konkretnej grupy docelowej.

Tylko czy To już byłaby reklama, czy selekcja klientów na podstawie ich preferencji i upodobań. I czy gdyby do tego doszło nie bylibyśmy już tylko bezwolnymi maszynami, które tylko kupują to, co zostanie im zaprezentowane w danym momencie?